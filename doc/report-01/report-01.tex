\documentclass[UTF-8,twoside,cs4size,hyperref]{ctexart}
\usepackage[dvipsnames]{xcolor}
\usepackage{amsmath}
\usepackage{amssymb}
\usepackage{geometry}
\usepackage{listings}
\usepackage{setspace}
\usepackage{xeCJK}
\usepackage{ulem}
\usepackage{pstricks}
\usepackage{pstricks-add}
\usepackage{bm}
\usepackage{mathtools}
\usepackage{breqn}
\usepackage{mathrsfs}
\usepackage{esint}
\usepackage{textcomp}
\usepackage{upgreek}
\usepackage{pifont}
\usepackage{tikz}
\usepackage{circuitikz}
\usepackage{caption}
\usepackage{tabularx}
\usepackage{array}
\usepackage{pgfplots}
\usepackage{multirow}
\usepackage{pgfplotstable}
\usepackage{mhchem}
\usepackage{longtable}
\usepackage[cache=false]{minted}
\usepackage{ragged2e}

\hypersetup{colorlinks=true,linkcolor=NavyBlue}

\newcolumntype{Y}{>{\centering\arraybackslash}X}
\geometry{a4paper,centering,top=2.54cm,bottom=2.54cm,left=2cm,right=2cm}
\graphicspath{{figures/}}
\pagestyle{plain}
\captionsetup{font=small}

%\CTEXsetup[name={,.}]{section}
\CTEXsetup[format={\raggedright\heiti\noindent\zihao{3}},numberformat={\bfseries}]{section}
\CTEXsetup[format={\raggedright\heiti\zihao{-3}\quad},numberformat={\bfseries}]{subsection}
\CTEXsetup[format={\raggedright\heiti\qquad\zihao{4}},numberformat={\bfseries}]{subsubsection}
\CTEXsetup[format={\raggedright\heiti\qquad},numberformat={\bfseries}]{paragraph}
\renewcommand\thefootnote{\ding{\numexpr171+\value{footnote}}}

\setstretch{1.5}

\setCJKfamilyfont{boldsong}[AutoFakeBold = {2.17}]{SimSun}
\newcommand*{\boldsong}{\CJKfamily{boldsong}}
%\DeclareMathOperator\dif{d\!}
\newcommand*{\me}{\mathop{}\!\mathrm{e}}
\newcommand*{\mpar}{\mathop{}\!\partial}
\newcommand*{\dif}{\mathop{}\!\mathrm{d}}
\newcommand*{\tab}{\indent}
\newcommand*{\mcelsius}{\mathop{}\!{^\circ}\mathrm{C}}
\renewcommand*{\Im}{\mathrm{Im}\,}

\setcounter{secnumdepth}{4}
\setcounter{tocdepth}{4}

\renewcommand\arraystretch{1.5}

\newfontfamily{\cascadia}{Cascadia Mono}

\usemintedstyle{manni}
\def\inlinecode#1{\colorbox{WhiteSmoke}{{\fontencoding{T1}\fontfamily{zi4}\fontsize{9pt}{9pt}\selectfont #1}}}
\newenvironment{code}[1]
{\VerbatimEnvironment
 \setstretch{1.2}
 \begin{minted}[obeytabs=false, linenos, breaklines=true, numbersep=-10pt]{#1}}
{\end{minted}
 \vspace{-0.3cm}
 \setstretch{1.5}}

%\setminted{style=manni,fontsize=\small,breaklines=true}


\tikzset{
	snode/.style ={
	    circle,
		inner sep=0pt,
		text width=10mm,
		draw = black,
		align=center
	}
}

\lstset{
	backgroundcolor=\color[RGB]{245,245,245},
	keywordstyle=\color{blue}\bfseries,
	basicstyle=\footnotesize\ttfamily,
	commentstyle=\itshape\color{olive},
	numberstyle=\ttfamily,
	stringstyle=\itshape\color{orange},
	tabsize=4,
	breaklines=true
}

\begin{document}
	\begin{center}
		\heiti\zihao{-2}
		数据库系统实验\textbf{2}设计报告
	\end{center}

	\begin{table*}[!h]
        \raggedleft
        \begin{tabular}{rc}
            高梓源 & 2019K8009929026\\
            肖展琪 & 2019K8009926014\\
            桂庭辉 & 2019K8009929019
        \end{tabular}
    \end{table*}
    
    \section{模型设计}
    \subsection{\textbf{ER}图与范式分析}
    
    \begin{figure}[!h]
        \centering
        \includegraphics*[width=\textwidth]{RE_2.pdf}
        \caption{火车订票系统ER图}
    \end{figure}
    
    \subsubsection{\textbf{users}}\label{sssec:users}
    
    候选键:uid,user\_name,tel\_num。作为非键属性的password与email没有其他的依赖关系。故users满足BCNF。
    
    passengers中候选键为pid,函数依赖关系仅有pid$\to$real\_name,故passengers满足BCNF。
    
    admin中候选键为aid,函数依赖关系仅有aid$\to$authority authentication,故admin满足BCNF。
    
    \begin{code}{sql}
    create table if not exists users (
        u_uid       serial primary key,
        u_user_name varchar(20) unique,
        u_password  varchar(20) not null,
        u_email     varchar(20) not null,
        u_tel_num   integer[11] unique
    );
    
    create table if not exists passengers (
        p_pid   integer not null,
        p_real_name varchar(20) not null,
        primary key (p_pid),
        foreign key (p_pid) references users (u_uid)
    );
    
    create table if not exists admin (
        a_aid            integer not null,
        a_authentication varchar(20) not null ,
        a_authority      admin_authority not null ,
        primary key (a_aid),
        foreign key (a_aid) references users (u_uid)
    );
    \end{code}
    
    \subsubsection{\textbf{orders}}
    
    主键为oid,函数依赖关系仅有由oid确定其他非键属性,故orders满足BCNF。
    
    \begin{code}{sql}
    create table if not exists orders (
        o_oid           serial primary key,
        o_uid           integer,
        o_train_id      integer      not null,
        o_date          date         not null,
        o_start_station integer      not null,
        o_end_station   integer      not null,
        o_seat_type     seat_type    not null,
        o_seat_id       integer      not null,
        o_status        order_status not null,
        o_effect_time   timestamp    not null,
        foreign key (o_uid) references users (u_uid),
        foreign key (o_train_id) references train (t_train_id),
        foreign key (o_start_station) references station_list (s_station_id),
        foreign key (o_end_station) references station_list (s_station_id)
    );
    \end{code}
    
    \subsubsection{\textbf{train}}\label{sssec:train}
    
    候选键包括train\_id与train\_name,二者均可确定train\_type,没有非平凡的依赖关系,故train满足BCNF。
    
    \begin{code}{sql}
    create table if not exists train (
        t_train_id   serial primary key,
        t_train_type varchar(1)  not null,
        t_train_name varchar(10) not null
    );
    \end{code}
    
    \subsubsection{\textbf{city}}
    
    主键为city\_id,其可唯一确定city\_name。此处引入的reach\_table,用于记录城市间在本实验需求下的可达关系,可通过train、train\_full\_info、station\_list等信息推导得到,在存储上构成了一定冗余,但用于实现需求5时有助于查找两地间车次的时间开销,后文讨论需求时会详细讨论。除reach\_table外该表满足BCNF。
    
    \begin{code}{sql}
    create table if not exists city (
        c_city_id     integer primary key,
        c_city_name   varchar(20) not null,
        c_reach_table boolean[]
    );
    \end{code}
    
    %TODO:city\_train
    
    \subsubsection{\textbf{station\_list}}\label{sssec:station_list}
    
    候选键包括station\_id,station\_name,二者均可确定city\_id,没有非平凡的依赖关系,故station\_list满足BCNF。
    
    \begin{code}{sql}
    create table if not exists station_list (
        s_station_id      serial primary key,
        s_station_name    varchar(20) not null,
        s_station_city_id integer     not null,
        foreign key (s_station_city_id) references city (c_city_id)
    );
    \end{code}
    
    \subsubsection{\textbf{train\_full\_info}}\label{sssec:tfi}
    
    该表记录每次列车每经停站的信息,主键为(train\_id, station\_id),每个条目中其他信息可由该二元组确定。由于存在跨天运行的列车,故而无法仅通过train\_id与arrive\_time或leave\_time确定所有信息。故train\_full\_info满足BCNF。
    
    在获取数据时,每车次列车初始域day\_from\_departure置0,列车运行时间每次跨越午夜零点则域day\_from\_departure自增1,由此记录列车准确的运行历时与进出站时间。
    
    \begin{code}{sql}
    create table if not exists train_full_info (
        tfi_train_id            integer,
        tfi_station_id          integer,
        tfi_station_order       integer          not null,
        tfi_arrive_time         time             not null,
        tfi_leave_time          time             not null,
        tfi_day_from_departure  integer          not null,
        tfi_distance            integer          not null,
        tfi_price               decimal(5, 1)[7] not null default array [0.0, 0.0, 0.0, 0.0, 0.0, 0.0, 0.0],
        primary key (tfi_train_id, tfi_station_id),
        foreign key (tfi_train_id) references train (t_train_id),
        foreign key (tfi_station_id) references station_list (s_station_id)
    );
    \end{code}
    
    \subsubsection{\textbf{station\_tickets}}\label{sssec:station_tickets}
    
    该表记录某站某日经停的某车各类型的余票,主键为(train\_id, station\_id, date),没有非平凡的依赖关系,station\_tickets满足BCNF。
    
    \begin{code}{sql}
    create table if not exists station_tickets (
        stt_station_id integer,
        stt_train_id   integer,
        stt_date       date       not null,
        stt_num        integer[7] not null default array [5, 5, 5, 5, 5, 5, 5],
        primary key (stt_station_id, stt_train_id),
        foreign key (stt_station_id) references station_list (s_station_id),
        foreign key (stt_train_id) references train (t_train_id),
        foreign key (stt_station_id, stt_train_id) references train_full_info (tfi_station_id, tfi_train_id)
    );
    \end{code}
    
    \subsection{关系模式}
    
    参考TPCH文档可画出如下关系模式图:
    
    \begin{figure}[!h]
        \centering
        \includegraphics*[width=\textwidth]{entity_relationship_diagram.pdf}
        \caption{火车订票系统关系模式图}
    \end{figure}
    
    具体的关系模式可见上一小节中的 \mintinline{sql}{create table} 语句,其中涉及到的自定义枚举类型补充如下:
    
    \begin{code}{sql}
    create type seat_type as enum ('YZ', 'RZ', 'YW_S', 'YW_Z', 'YW_X', 'RW_S', 'RW_X');
    create type order_status as enum ('COMPLETE', 'PRE_ORDERED', 'CANCELED');
    create type admin_authority as enum ('ALL');
    \end{code}
    
    \section{需求实现}
    \subsection{需求\textbf{1 $\sim$ 3}:记录相关信息}
    
    需求1要求记录每车次列车相关信息,可通过 \ref{sssec:train}、\ref{sssec:station_list}、\ref{sssec:tfi} 中讨论的表记录。
    
    需求2要求记录列车座位情况,可通过 \ref{sssec:station_tickets} 中讨论的表与上述列车信息记录。
    
    需求3中对乘客信息的记录部分可通过 \ref{sssec:users} 中users与passengers表记录。注册与登录操作对应sql语句如下,其中对密码的加密工作交由前端php完成。
    
    \begin{code}{sql}
    /* user registration */
    /* @param: user_name */
    /*       : user_password */
    /*       : phone_num */
    /*       : user_email */
    /* @return: uid */
    begin
        select * into uid from insert_all_info_into__u__(user_name, user_password, phone_num, user_email);
    end;
    
    /* user login */
    /* @param: user_name */
    /*       : user_password */
    /* @return: uid */
    /*        : error */
    begin
        select * from query_uid_from_uname_password__u__(user_name, user_password) into uid, error;
    end;
    \end{code}
    
    上述代码中使用到的两个函数的sql实现如下:
    \begin{code}{sql}
    /* query_uid_from_uname_password__u__ */
    /* @param: user_name */
    /*       : user_password */
    /* @return: uid */
    /*        : error_type */
    begin
        if (select * from users where u_user_name = user_name) is null then
            uid := 0;
            error := 'ERROR_NOT_FOUND_UNAME';
        else
            select u_uid into uid from users where u_user_name = user_name and u_password = user_password;
            if uid is null then
                uid := 0;
                error := 'ERROR_NOT_CORRECT_PASSWORD';
            else
                error := 'NO_ERROR';
            end if;
        end if;
    end;
    
    /* insert_all_info_into__u__ */
    /* @param: user_name */
    /*       : user_password */
    /*       : phone_num */
    /*       : user_email */
    /* @return: uid */
    /*        : err */
    begin
        if (select * from users where u_user_name = user_name) is not null then
            uid := 0;
            err := 'ERROR_DUPLICATE_UNAME';
        else
            if (select * from users where u_tel_num = phone_num) is not null then
                uid := 0;
                err := 'ERROR_DUPLICATE_U_TEL_NUM';
            else
                insert into users (u_user_name, u_password, u_email, u_tel_num)
                    values (user_name, user_password, user_email, phone_num);
                select currval(pg_get_serial_sequence('users', 'u_uid')) into uid;
                err := 'NO_ERROR';
            end if;
        end if;
    end;
    \end{code}
    
    \subsection{需求\textbf{4}:查询具体车次}
    
    需求1中已根据车次记录列车信息,根据表train、station\_list、train\_full\_info查询信息并组织正确的输出形式即可。
    
    \begin{code}{sql}
    /*@param: train_name  */
    /*      : q_date */
    begin
        select query_train_id_from_name__t__(train_name) into train_id;
        select query_start_time_from_id__tfi__(train_id) into start_time;
        return query select tfi_station_order as station_order,
                            s_station_name as station,
                            s_station_id as station_id,
                            c_city_name as city_name,
                            c_city_id as city_id,
                            tfi_arrive_time as arrive_time,
                            tfi_leave_time as leave_time,
                            (select * from get_actual_interval_bt_time(tfi_arrive_time, tfi_leave_time, 0)) as stay_time,
                            (select * from get_actual_interval_bt_time(start_time, tfi_arrive_time, (select query_day_from_departure_from_id__tfi__(train_id, tfi_station_id)))) as durance,
                            tfi_distance as distance,
                            tfi_price as seat_price,
                            stt_num as seat_num
                         from train_full_info tfi
                                  left join station_list s on tfi.tfi_station_id = s.s_station_id
                                  left join city c on s.s_station_city_id = c.c_city_id
                                  left join train t on tfi.tfi_train_id = t.t_train_id
                                  left join station_tickets stt on tfi.tfi_station_id = stt.stt_station_id
                         where t_train_id = train_id
                           and stt.stt_date = q_date;
    end;
    
    /* get_actual_interval_bt_time */
    /* @param: start_time */
    /*       : end_time */
    /*       : days_added */
    /* @return: actual_interval */
    begin
        if days_added = 0 and start_time > end_time then
            actual_interval := interval '24 hours' + end_time - start_time;
        else
    	    actual_interval := (days_added || 'days')::interval + end_time - start_time;
        end if;    
    end
    \end{code}
    
    此处使用的函数 \mintinline{sql}{query_train_id_from_name__t__} 等均为简单的匹配查找,此处与后文均不详述此类函数。
    
    \subsection{需求\textbf{5、6}:查询两地之间的车次与返程信息}
    
    需求6相对需求5仅交换起始城市,可交由前端完成,sql逻辑复用需求5即可。
    
    直达列车的查询逻辑较为简单,对起始站点查询此处离开的列车,遍历这些可能的列车,通过输入的出发时间过滤掉发车时间早于出发时间的列车,通过车次与城市即可确定可能的终点站,而后则是余票与价格的计算。余票与车次座位的设计维护在需求7中说明。
    
    \begin{code}{sql}
    /* @func: get_train_bt_cities_directly */
    /* @param: from_city_id */
    /*       : to_city_id */
    /*       : q_date */
    /*       : q_time */
    begin
        select check_reach_table(from_city_id, to_city_id) into city_reachable;
        if city_reachable then
            train_id_list := array(
                    select from_city_train.ct_train_id
                        from city_train from_city_train
                                 join city_train to_city_train on from_city_train.ct_train_id = to_city_train.ct_train_id
                        where (select get_ct_priority(from_city_id, from_city_train.ct_train_id)) <
                              (select get_ct_priority(to_city_id, to_city_train.ct_train_id))
                );
            <<scan_train_list>>
            foreach train_idi in array train_id_list
                loop
                -- 2 ways of accomplishment --
                -- leave station --
                    select get_station_id_from_cid_tid(from_city_id, train_idi) into station_leave_id;
                    select query_station_name_from_id__s__(station_leave_id) into station_leave_name;
                    select q_all_info_leave.leave_time,
                           q_all_info_leave.day_from_departure,
                           q_all_info_leave.distance,
                           q_all_info_leave.price
                        into station_leave_time, station_leave_day, station_leave_distance, station_arrive_price
                        from query_train_all_info_from_tid_sid__tfi__(train_idi, station_leave_id) q_all_info_leave;
                    -- check time --
                    if station_leave_time < q_time then
                        continue scan_train_list;
                    end if;
                    select query_train_name_from_id__t__(train_idi) into train_namei;
                    -- arrive station --
                    select get_station_id_from_cid_tid(to_city_id, train_idi) into station_arrive_id;
                    select query_station_name_from_id__s__(station_arrive_id) into station_arrive_name;
                    select q_all_info_arrive.leave_time,
                           q_all_info_arrive.day_from_departure,
                           q_all_info_arrive.distance,
                           q_all_info_arrive.price
                        into station_arrive_time, station_arrive_day, station_arrive_distance, station_leave_price
                        from query_train_all_info_from_tid_sid__tfi__(train_idi, station_arrive_id) q_all_info_arrive;
                    -- seats and price calculation --
                    for seat_i in 1..7
                        loop
                            select array_set(station_arrive_price, station_arrive_price[seat_i],
                                             station_arrive_price[seat_i] - station_leave_price[seat_i])
                                into res_price;
                        end loop;
                    select get_min_seat.seat_num
                        into seat_nums
                        from get_min_seats(train_idi, q_date, station_leave_id, station_arrive_id,
                                           array ['YZ', 'RZ', 'YW_S', 'YW_Z', 'YW_X', 'RW_S', 'RW_X']) get_min_seat;
                    -- return row --
                    for r in
                        select train_namei as train_name,
                               train_idi as train_id,
                               station_leave_name as station_from_name,
                               station_leave_id as station_from_id,
                               station_arrive_name as station_to_name,
                               station_arrive_id as station_to_id,
                               station_leave_time as leave_time,
                               station_arrive_time as arrive_time,
                               (station_arrive_day - station_leave_day || 'days')::interval + station_arrive_time -
                               station_leave_time as durance,
                               station_arrive_distance - station_leave_distance as distance,
                               res_price as seat_price,
                               seat_nums as seat_nums,
                               false as transfer_first,
                               false as transfer_late
                        loop
                            return next r;
                        end loop;
                end loop;
        end if;
        return;
    end;
    \end{code}
    
    对于换乘一次的情况,对于出发城市,查找其可直达的所有城市,遍历这些城市以它们为换乘城市,查找到目的城市的可能列车(即查询从换乘城市到目的城市的直达列车),事先通过city表中reach\_table域记录城市间的可达与否可以在遍历换乘城市时避免与目的城市完全不可达的城市作为换乘城市时的查询操作(即上文函数\mintinline{sql}{get_train_bt_cities_directly}开头的\mintinline{sql}{city_reachable}检查),减小开销。
    
    具体实现上类似BFS,通过队列记录可能的换乘城市,遍历所有可能的情况直到清空队列。
    
    以下代码指定\mintinline{sql}{query_transfer}为真时查询换乘一次的结果,为否则调用上文函数查询直达列车。
    
    \begin{code}{sql}
    /* @param: city_from */
    /*       : city_to */
    /*       : q_date */
    /*       : q_time */
    /*       : query_transfer */
    begin
        select query_city_id_from_name__c__(city_from) into from_city_id;
        select query_city_id_from_name__c__(city_to) into to_city_id;
        select check_reach_table(from_city_id, to_city_id) into city_reachable;
        if city_reachable then
            if not query_transfer then
                for r in
                    select * from get_train_bt_cities_directly(from_city_id, to_city_id, q_date, q_time)
                    loop
                        return next r;
                    end loop;
            else
                -- first set of transfer trains must be ones passing from city --
                -- so outside loop --
                passing_trains := array(select query_train_id_list_from_cid__ct__(from_city_id));
                while (select array_length(src_city, 1)) > 0 and (select array_position(src_city, to_city_id)) is not null
                    loop
                        select array_length(src_city, 1) into current_level_city_num;
                        for city_i in 1..current_level_city_num
                            loop
                                neighbour_city := array(select get_ct_next_city_list(src_city[1], passing_trains));
                                src_city := array(select array_cat(src_city, neighbour_city));
                                -- initially from_city_id was in src_city --
                                -- so remove it first because we have dealt with it --
                                src_city := array(select array_remove_elem(src_city, 1));
                                -- then src_city[1] is middle city to transfer --
                                if (select array_length(src_city, 1)) > 1 then
                                    -- ... --
                                end if;
                            end loop;
                    end loop;
                return;
            end if;
        end if;
    end;
    \end{code}
    
    对于每个可能的换乘城市,遍历出发城市到其的每次直达列车,查询其到目的城市的每次直达列车,根据上游列车与下游列车是否在同站换乘检查不同的时间要求是否满足。
    
    \begin{code}{sql}
    if (select array_length(src_city, 1)) > 1 then
        for r in
            (select *
                    from get_train_bt_cities_directly(from_city_id, src_city[1], q_date, q_time))
            loop
                for j in
                    (select *
                            from get_train_bt_cities_directly(src_city[1], to_city_id, q_date, q_time + r.durance + interval '1 hour'))
                    loop
                        if r.station_to_id = j.station_from_id then
                            select *
                                from get_actual_interval_bt_time(r.arrive_time, j.leave_time, 0)
                                into transfer_interval;
                            if transfer_interval >= interval '1 hour'
                                and transfer_interval <= interval '4 hours'
                            then
                                return next r;
                                return next j;
                            end if;
                        else
                            if transfer_interval >= interval '2 hours'
                                and transfer_interval <= interval '4 hours'
                            then
                                return next r;
                                return next j;
                            end if;
                        end if;
                    end loop;
            end loop;
    end if;
    \end{code}
        
    \subsection{需求\textbf{7}:预订车次座位}
    
    余票与车次座位的设计上,我们在station\_tickets表中记录某天某车在某站的各类型余票数,每当乘客购买某天某一区间车票,则将该天该车该区间内(包括左端不包括右端)所有对应类型余票数减1。查询时对某车某天某区间的某类型余票数则为该区间内每站该类型余票数的最小值。获取座位时先查找区间内合法的座位,其后更新station\_tickets表内容。
    
    \begin{code}{sql}
    /* try_occupy_seats */
    /* @param: train_id */
    /*       : order_date */
    /*       : station_from_id */
    /*       : station_to_id */
    /*       : seat_type */
    /*       : seat_num */
    /* @return: succeed */
    /*        : left_seat */
    begin
        select query_station_order_from_tid_sid__tfi__(train_id, station_from_id) into station_start_order;
        station_order_ptr = station_start_order;
        select query_station_order_from_tid_sid__tfi__(train_id, station_to_id) into station_end_order;
        -- find min_seat --
        select get_min_seat.seat_num
            into min_seat
            from get_min_seats(train_id, order_date, station_from_id, station_to_id, array [seat_type]) get_min_seat
            where in_order = 1;
        -- check satisfiability --
        if min_seat < seat_num then
            succeed := false;
            left_seat := min_seat;
        else
            succeed := true;
            left_seat := 5 - min_seat;
            -- second loop, update station tickets --
            while station_order_ptr != station_end_order
                loop
                    update station_tickets
                    set stt_num = (select array_set(stt_num, seat_type, stt_num[seat_type] - seat_num))
                        where stt_train_id = train_id
                          and stt_station_id = station_id_ptr
                          and stt_date = order_date;
                    station_order_ptr := station_order_ptr + 1;
                    select query_station_id_from_tid_so__tfi__(train_id, station_order_ptr) into station_id_ptr;
                end loop;
        end if;
    end;
    \end{code}
    
    用户预定座位时,在确定区间车次日期等信息申请订单后即暂时拥有座位(如有),此时记订单状态为\texttt{PRE_ORDERED},点击确认后将订单置为\texttt{ORDERED}。有关订单状态的维护将在需求8内讨论。
    
    \begin{code}{sql}
    /* pre_order_train */
    /* @param: train_id */
    /*       : station_from_id */
    /*       : station_to_id */
    /*       : seat_type */
    /*       : seat_num */
    /*       : order_date */
    /* @return: succeed */
    /*        : seat_id */
    /*        : order_id */
    begin
        select succeed, left_seat
            into succeed, seat_id
            from try_occupy_seats(train_id, order_date, station_from_id, station_to_id, seat_type, seat_num);
        if succeed then
            insert into orders (o_train_id, o_date, o_start_station, o_end_station, o_seat_type, o_seat_id,
                                o_status, o_effect_time)
            select train_id,
                   order_date,
                   station_from_id,
                   station_to_id,
                   seat_type,
                   seat_id,
                   'PRE_ORDERED',
                   now();
            select currval(pg_get_serial_sequence('orders', 'o_oid')) into order_id;
        else
            order_id := 0;
        end if;
    end;
    
    /* order_train_seats */
    /* @param: order_id */
    /*       : uid */
    /* @return: succeed */
    begin
        update orders
        set (o_uid, o_status) = (uid, 'ORDERED')
            where o_oid = order_id;
    end;
    \end{code}
    
    \subsection{需求\textbf{8}:查询订单和删除订单}
    
    订单相关信息记录在orders表内,给定用户、出发日期范围等信息后即可查询获得。
    
    \begin{code}{sql}
    select o_oid as order_id,
        t_train_name as train_name,
        o_train_id as train_id,
        s_start.s_station_name as station_leave,
        o_start_station as station_id,
        s_arrive.s_station_name as station_arrive,
        tfi_start.tfi_leave_time as start_time,
        tfi_end.tfi_arrive_time as arrive_time,
        (select * from get_actual_interval_bt_time(tfi_start.tfi_leave_time, tfi_end.tfi_arrive_time,
            tfi_start.tfi_day_from_departure - tfi_end.tfi_day_from_departure)) as durance,
        tfi_end.tfi_distance - tfi_start.tfi_distance as distance,
        o_seat_type as seat_type,
        o_seat_id as seat_id,
        o_status as status,
        tfi_end.tfi_price[o_seat_type] - tfi_start.tfi_price[o_seat_type] + 5 as price
        from orders
                left join station_list s_start on orders.o_start_station = s_start.s_station_id
                left join station_list s_arrive on orders.o_end_station = s_arrive.s_station_id
                left join train on orders.o_train_id = train.t_train_id
                left join train_full_info tfi_start on s_start.s_station_id = tfi_start.tfi_station_id
            and orders.o_train_id = tfi_start.tfi_train_id
                left join train_full_info tfi_end on s_start.s_station_id = tfi_end.tfi_station_id
            and orders.o_train_id = tfi_end.tfi_train_id
        where o_uid = uid
        and o_date >= start_query_date
        and o_date <= end_query_date;
    \end{code}
    
    我们将订单的状态分为三种:取消、预约、确认。用户发起申请后确认订单前订单为预约态,此时对订单的操作仅有点击确认完成订单进入确认态,30分钟内未确认的订单将会自动老化删除。确认态的订单可被手动取消。确认操作的sql语句在需求7中已经给出,以下给出取消与老化操作的sql语句:
    \begin{code}{sql}
    /* user_cancel sql */
    /* @param: order_id */
    /* @return: succeed */
    begin
        select o_train_id, o_date, o_start_station, o_end_station, o_seat_type
            into train_id, order_date, start_station, end_station, seat_type
            from orders
            where o_oid = order_id
              and o_status = 'COMPLETE';
        select release_seats(train_id, order_date, start_station, end_station, seat_type, 1);
        update orders
        set o_status = 'CANCELED'
            where o_oid = order_id;
    end;
    
    /* remove_outdated_order */
    delete
        from orders
        where (select * from get_actual_interval_bt_time(orders.o_effect_time, now(), 0))
                  > interval '30 minutes'
            and orders.o_status = 'PRE_ORDERED';
    \end{code}
    
    \subsection{需求\textbf{9}:管理员}
    
    管理员相关需求中,查看每个用户订单一项可复用需求8中查看订单的实现。查询当前注册用户列表的实现也较简单,对passengers表\footnote{我们将所有用户users分为passengers和admin两类,约定管理员的查询功能为查询所有passengers}查询即可。
    \begin{code}{sql}
    select p_pid as uid,
           u_user_name as uname,
           array(
                select o_oid from orders where o_uid = p_pid
           ) as orders
        from passengers
            left join users u on passengers.p_pid = u.u_uid;
    \end{code}
    
    对总订单数、总票价、最热点车次排序的查询涉及到订单的取消态,由于我们额外引入了预约态,此处约定这三处查询仅包括已确认的订单,不统计取消订单与已申请未确认的订单。
    
    \begin{code}{sql}
    -- 2 views to select top 10 train --
    create or replace view top_10_train_tickets(train_id, count_num)
    as
    select o_train_id as train_id, count(*) as count_num
        from orders
        where o_status = 'COMPLETE'
        group by o_train_id
        order by count_num
        limit 10;
    
    create or replace view top_10_train_ids(train_id)
    as
    select train_id
        from top_10_train_tickets;
    
        
    /* admin_query_orders */
    /* @return: total_order_num */
    /*        : total_price */
    /*        : hot_trains */
    begin
        select count(*),
               sum(tfi_end.tfi_price[o_seat_type] - tfi_start.tfi_price[o_seat_type] + 5)
            into total_order_num, total_price
            from orders
                     left join train_full_info tfi_start on o_start_station = tfi_start.tfi_station_id
                and orders.o_train_id = tfi_start.tfi_train_id
                     left join train_full_info tfi_end on o_end_station = tfi_end.tfi_station_id
                and orders.o_train_id = tfi_end.tfi_train_id
                and orders.o_status = 'COMPLETE';
        hot_trains := array(select t_train_name
                                from train
                                         left join top_10_train_ids on train.t_train_id = top_10_train_ids.train_id);
    end;
    \end{code}
    
    \section*{成员分工}
    
    高梓源:
    
    肖展琪:
    
    桂庭辉:
    
\end{document}